\documentclass[]{book}
\usepackage{lmodern}
\usepackage{amssymb,amsmath}
\usepackage{ifxetex,ifluatex}
\usepackage{fixltx2e} % provides \textsubscript
\ifnum 0\ifxetex 1\fi\ifluatex 1\fi=0 % if pdftex
  \usepackage[T1]{fontenc}
  \usepackage[utf8]{inputenc}
\else % if luatex or xelatex
  \ifxetex
    \usepackage{mathspec}
  \else
    \usepackage{fontspec}
  \fi
  \defaultfontfeatures{Ligatures=TeX,Scale=MatchLowercase}
\fi
% use upquote if available, for straight quotes in verbatim environments
\IfFileExists{upquote.sty}{\usepackage{upquote}}{}
% use microtype if available
\IfFileExists{microtype.sty}{%
\usepackage{microtype}
\UseMicrotypeSet[protrusion]{basicmath} % disable protrusion for tt fonts
}{}
\usepackage{hyperref}
\hypersetup{unicode=true,
            pdftitle={Basics of Health Intelligent Data Analysis},
            pdfauthor={Bruno Lima, Cátia Redondo},
            pdfborder={0 0 0},
            breaklinks=true}
\urlstyle{same}  % don't use monospace font for urls
\usepackage{natbib}
\bibliographystyle{apalike}
\usepackage{color}
\usepackage{fancyvrb}
\newcommand{\VerbBar}{|}
\newcommand{\VERB}{\Verb[commandchars=\\\{\}]}
\DefineVerbatimEnvironment{Highlighting}{Verbatim}{commandchars=\\\{\}}
% Add ',fontsize=\small' for more characters per line
\usepackage{framed}
\definecolor{shadecolor}{RGB}{248,248,248}
\newenvironment{Shaded}{\begin{snugshade}}{\end{snugshade}}
\newcommand{\KeywordTok}[1]{\textcolor[rgb]{0.13,0.29,0.53}{\textbf{#1}}}
\newcommand{\DataTypeTok}[1]{\textcolor[rgb]{0.13,0.29,0.53}{#1}}
\newcommand{\DecValTok}[1]{\textcolor[rgb]{0.00,0.00,0.81}{#1}}
\newcommand{\BaseNTok}[1]{\textcolor[rgb]{0.00,0.00,0.81}{#1}}
\newcommand{\FloatTok}[1]{\textcolor[rgb]{0.00,0.00,0.81}{#1}}
\newcommand{\ConstantTok}[1]{\textcolor[rgb]{0.00,0.00,0.00}{#1}}
\newcommand{\CharTok}[1]{\textcolor[rgb]{0.31,0.60,0.02}{#1}}
\newcommand{\SpecialCharTok}[1]{\textcolor[rgb]{0.00,0.00,0.00}{#1}}
\newcommand{\StringTok}[1]{\textcolor[rgb]{0.31,0.60,0.02}{#1}}
\newcommand{\VerbatimStringTok}[1]{\textcolor[rgb]{0.31,0.60,0.02}{#1}}
\newcommand{\SpecialStringTok}[1]{\textcolor[rgb]{0.31,0.60,0.02}{#1}}
\newcommand{\ImportTok}[1]{#1}
\newcommand{\CommentTok}[1]{\textcolor[rgb]{0.56,0.35,0.01}{\textit{#1}}}
\newcommand{\DocumentationTok}[1]{\textcolor[rgb]{0.56,0.35,0.01}{\textbf{\textit{#1}}}}
\newcommand{\AnnotationTok}[1]{\textcolor[rgb]{0.56,0.35,0.01}{\textbf{\textit{#1}}}}
\newcommand{\CommentVarTok}[1]{\textcolor[rgb]{0.56,0.35,0.01}{\textbf{\textit{#1}}}}
\newcommand{\OtherTok}[1]{\textcolor[rgb]{0.56,0.35,0.01}{#1}}
\newcommand{\FunctionTok}[1]{\textcolor[rgb]{0.00,0.00,0.00}{#1}}
\newcommand{\VariableTok}[1]{\textcolor[rgb]{0.00,0.00,0.00}{#1}}
\newcommand{\ControlFlowTok}[1]{\textcolor[rgb]{0.13,0.29,0.53}{\textbf{#1}}}
\newcommand{\OperatorTok}[1]{\textcolor[rgb]{0.81,0.36,0.00}{\textbf{#1}}}
\newcommand{\BuiltInTok}[1]{#1}
\newcommand{\ExtensionTok}[1]{#1}
\newcommand{\PreprocessorTok}[1]{\textcolor[rgb]{0.56,0.35,0.01}{\textit{#1}}}
\newcommand{\AttributeTok}[1]{\textcolor[rgb]{0.77,0.63,0.00}{#1}}
\newcommand{\RegionMarkerTok}[1]{#1}
\newcommand{\InformationTok}[1]{\textcolor[rgb]{0.56,0.35,0.01}{\textbf{\textit{#1}}}}
\newcommand{\WarningTok}[1]{\textcolor[rgb]{0.56,0.35,0.01}{\textbf{\textit{#1}}}}
\newcommand{\AlertTok}[1]{\textcolor[rgb]{0.94,0.16,0.16}{#1}}
\newcommand{\ErrorTok}[1]{\textcolor[rgb]{0.64,0.00,0.00}{\textbf{#1}}}
\newcommand{\NormalTok}[1]{#1}
\usepackage{longtable,booktabs}
\usepackage{graphicx,grffile}
\makeatletter
\def\maxwidth{\ifdim\Gin@nat@width>\linewidth\linewidth\else\Gin@nat@width\fi}
\def\maxheight{\ifdim\Gin@nat@height>\textheight\textheight\else\Gin@nat@height\fi}
\makeatother
% Scale images if necessary, so that they will not overflow the page
% margins by default, and it is still possible to overwrite the defaults
% using explicit options in \includegraphics[width, height, ...]{}
\setkeys{Gin}{width=\maxwidth,height=\maxheight,keepaspectratio}
\IfFileExists{parskip.sty}{%
\usepackage{parskip}
}{% else
\setlength{\parindent}{0pt}
\setlength{\parskip}{6pt plus 2pt minus 1pt}
}
\setlength{\emergencystretch}{3em}  % prevent overfull lines
\providecommand{\tightlist}{%
  \setlength{\itemsep}{0pt}\setlength{\parskip}{0pt}}
\setcounter{secnumdepth}{5}
% Redefines (sub)paragraphs to behave more like sections
\ifx\paragraph\undefined\else
\let\oldparagraph\paragraph
\renewcommand{\paragraph}[1]{\oldparagraph{#1}\mbox{}}
\fi
\ifx\subparagraph\undefined\else
\let\oldsubparagraph\subparagraph
\renewcommand{\subparagraph}[1]{\oldsubparagraph{#1}\mbox{}}
\fi

%%% Use protect on footnotes to avoid problems with footnotes in titles
\let\rmarkdownfootnote\footnote%
\def\footnote{\protect\rmarkdownfootnote}

%%% Change title format to be more compact
\usepackage{titling}

% Create subtitle command for use in maketitle
\providecommand{\subtitle}[1]{
  \posttitle{
    \begin{center}\large#1\end{center}
    }
}

\setlength{\droptitle}{-2em}

  \title{Basics of Health Intelligent Data Analysis}
    \pretitle{\vspace{\droptitle}\centering\huge}
  \posttitle{\par}
    \author{Bruno Lima, Cátia Redondo}
    \preauthor{\centering\large\emph}
  \postauthor{\par}
      \predate{\centering\large\emph}
  \postdate{\par}
    \date{2019-11-02}

\usepackage{booktabs}

\begin{document}
\maketitle

{
\setcounter{tocdepth}{1}
\tableofcontents
}
\chapter*{}\label{section}
\addcontentsline{toc}{chapter}{}

\includegraphics{images/heads.png}

\chapter*{Preface}\label{preface}
\addcontentsline{toc}{chapter}{Preface}

This is a \emph{book} written in \textbf{Markdown} through
\emph{RStudio}.

The \textbf{bookdown} package \citep{xie2015} can be installed from CRAN
or Github:

\begin{Shaded}
\begin{Highlighting}[]
\KeywordTok{install.packages}\NormalTok{(}\StringTok{"bookdown"}\NormalTok{)}
\CommentTok{# or the development version}
\CommentTok{# devtools::install_github("rstudio/bookdown")}
\end{Highlighting}
\end{Shaded}

To compile this example to PDF, you need XeLaTeX. You are recommended to
install TinyTeX (which includes XeLaTeX):
\url{https://yihui.name/tinytex/}.

In this \emph{book}, we will present our results for the work we made on
the subject of \textbf{Basics of Health Intelligent Data Analysis} from
the \textbf{HEADS} PhD programme.

An exhaustive explanation using the \textbf{bookdown} package
\citep{R-bookdown} can be found at bookdown: Authoring Books and
Technical Documents with R Markdown and this is only a sample book,
which was built on top of R Markdown and \textbf{knitr}

\chapter{Introduction}\label{intro}

A legend from the III century AC bring us Saints Cosmas and Damian, two
christian martyrs whom practiced medicine in the region of Aegeae,
actual southwest Turkey. They were Anarguroi, reluctant of asking for
money and, according to the legend or miracle, they were capable of
transplanting the leg of a dead Ethiopian to an amputated Individual.
Many years later, 17 centuries after that, at Boston in 1954, Doctor
Joseph Murray succeeded in transplanting a kidney to an individual. The
transplantation performed by the Southafrican surgeon, Christiaan
Barnard in 1967, the first successful heart transplantation ever
performed, was probably the one that made transplantation surgery
front-page prompting even more longings and promising dreams. Since
then, the technique has developed enormously.

Nowadays, organ transplantation is a medical practice performed
worldwide. If we take a look at that transplantation cartography it
reveals that the distribution of such technique, or clinical practice is
deeply unequal. Where economic resources exist where technology is
available transplantation surgery is almost routine. However, in many
places in the world is simply, and unfortunately, nothing but a dream.

In this study we aim to do a descritive analysis of deceased organ
donors used for transplantation in Portugal.

\chapter{Methods}\label{method}

Data used in this study was collected from IPST data base the register
for organ transplants performed in Portugal. Between 2016 and 2018, a
total of 891 deceased organ donors where registered. For each donor data
on transplantation date, region, sex, age, height, weight, cause of
death, Coordination Transplant Office (CTO) (table \ref{tab:ctos}), AB0
blood type, HLA typing, transplanted organs (kidney, heart, lung, leaver
and pancreas), kidney transplant unit (TxU) (table \ref{tab:txus}) and
diagnosis of death (brain death (BD) or cardiocirculatory death (CD)).

We performed a descriptive analysis in order to summarize deceased organ
donors characteristics.

\begin{table}

\caption{\label{tab:ctos}Coordination Transplant Offices}
\centering
\begin{tabular}[t]{lll}
\toprule
CTO & name & region\\
\midrule
CHP & Centro Hospitalar do Porto & North\\
CHSJ & Centro Hospitalar Sao Joao & North\\
CHUC & Centro Hospitalar Universidade de Coimbra & Center\\
CHLC & Centro Hospitalar Lisboa Central & South\\
CHLN & Centro Hopitalar Lisboa Norte & South\\
\bottomrule
\end{tabular}
\end{table}

In Portugal we have a total of 5 Coordination Transplant Offices (CTO)
responsables for the identification of potential donors, and 8 kidney
Transplant Units (TxU) divided by 3 regions.

\begin{table}

\caption{\label{tab:txus}Kidney Transplant Units}
\centering
\begin{tabular}[t]{l|l|l}
\hline
TxU & name & region\\
\hline
CHP & Centro Hospitalar do Porto & North\\
\hline
CHSJ & Centro Hospitalar Sao Joao & North\\
\hline
CHUC & Centro Hospitalar Universidade de Coimbra & Center\\
\hline
CHLC & Centro Hospitalar Lisboa Central & South\\
\hline
CHLN & Centro Hopitalar Lisboa Norte & South\\
\hline
CHLO & Centro Hospitalar Lisboa Norte & South\\
\hline
CVP & Cruz Vermelha Portuguesa & South\\
\hline
HGO & Hospital Garcia de Orta & South\\
\hline
\end{tabular}
\end{table}

All the analysis and graphic representations where performed in
\emph{RStudio}, an integrated development environment (IDE) for
\textbf{R} programming language.

\chapter{Results}\label{result}

Between 2016 and 2018, we have a total of 295, 308 and 288 donors
resopectively (table \ref{tab:dyear})

\begin{tabular}{r|r}
\hline
year & n\\
\hline
2016 & 295\\
\hline
2017 & 308\\
\hline
2018 & 288\\
\hline
\end{tabular}

\section{Donors}\label{donors}

\section{Transplants}\label{transplants}

\chapter{Discussion}\label{discus}

A discussion about our results.

\chapter{Final Words}\label{final-words}

We have finished a nice book.

\bibliography{book.bib,packages.bib}


\end{document}
